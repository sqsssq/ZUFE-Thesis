
% \fancypagestyle{abstract}{
%   % 页眉高度
%   \setlength{\headheight}{10pt}

%   % 页眉和页脚(页码)的格式设定
%   \fancyhf{}
%   \fancyhead[C]{\ziju{0.08}\songti\zihao{-5}{浙江财经大学本科生毕业论文(设计)}}

%   % 页眉分割线稍微粗一些
%   \renewcommand{\headrulewidth}{0.4pt}
% }

% \pagestyle{abstract}
% \topskip=0pt

% 中英文摘要章节
\zihao{-4}
\vspace*{-10mm}

\begin{center}
  \heiti\zihao{3}\textmd{\thesisTitle}
  \vspace{2mm}
  \ifhaveSubTitle
  \begin{spacing}{1.2}
    \sihao\selectfont{\textmd{\mystkaiti{-----}\thesisSubTitle}}
  \end{spacing}
  \fi
\end{center}

\vspace*{0mm}

{\let\clearpage\relax \chapter*{\textmd{}}}
% 加入目录
\addcontentsline{toc}{chapter}{摘~~~~要}
\setcounter{page}{1}

\vspace*{-12mm}

\setstretch{1.53}
\setlength{\parskip}{0em}

\textbf{\heiti\xiaosi{摘要:}}
% 中文摘要正文从这里开始-----------------------------------------------------------------
{\mystkaiti{
本文……生僻字:垚瑄。{\songti{正文默认字体为\textbackslash{}songti命令,与支持生僻字的{\mystsong{方正宋体}}略有区别,因此正文中宋体生僻字使用\textbackslash{}mystsong命令代替即可。如:\mystsong{垚瑄}}}。
\textcolor{blue}{摘要的内容要包括研究的目的、方法、结果和结论。计量单位一律换算成国际标准计量单位。除特殊情况外,数字一律用阿拉伯数字。中、英文摘要的内容应严格一致。}
}}

\vspace{1em}
\textbf{\heiti\xiaosi{关键词:}}
{\mystkaiti{
计算机科学与技术;算法;复杂度;深度学习;机器学习;可视化计算
}}
% \newpage

% 英文摘要章节
\vspace*{15mm}

\begin{spacing}{0.95}
  \centering
  \zihao{3}\textbf{\thesisTitleEN}
  \vspace{2mm}
  \ifhaveSubTitle
  \begin{spacing}{1.2}
    \sihao\selectfont{\textmd{{-----}\thesisSubTitleEN}}
  \end{spacing}
  \fi
\end{spacing}

\vspace*{0mm}

{\let\clearpage\relax \chapter*{\zihao{-3}\textmd{}\vskip -3bp}}
\addcontentsline{toc}{chapter}{Abstract}
\setcounter{page}{1}

\setstretch{1.53}
\setlength{\parskip}{0em}

\textbf{\heiti\xiaosi{Abstract: }}
% 英文摘要正文从这里开始-----------------------------------------------------------------
In order to study……


\textbf{\xiaosi{Key words:}}
Computer science and technology; Algorithm; Complexity; Deep learning; Machine learning; Visualization analysis
\newpage
