% 引言

\chapter{引~~~~言}
本文主要研究了……

本模板(继承ctexbook)支持的字体有:
\begin{itemize}
  \item \verb|\songti|{\songti :设置为宋体(默认中文字体),生僻字:垚瑄}
  \item \verb|\heiti|{\heiti :设置为黑体,生僻字:垚瑄}
  \item \verb|\kaishu|{\kaishu :设置为楷书,生僻字:垚瑄}
  \item \verb|\fangsong|{\fangsong :设置为仿宋,生僻字:垚瑄}
  %\item \verb|\lishu|{\lishu :设置为隶书(部分环境支持),生僻字:垚瑄}
\end{itemize}

本模板正文默认字体为宋体,命令为\verb|\songti|,有些系统或在线平台(Overleaf、TexPage等)自带的宋体可能不支持生僻字,此时需要使用本模板中支持生僻字的字体{\mystsong{方正宋体}},其命令为\verb|\mystsong|,使用方法为:\verb|{\mystsong{生僻字}}|。

\textcolor{blue}{生僻字:垚瑄。}若这行无法看到全部内容,则表示默认的字体不支持生僻字。

\textcolor{blue}{\mystsong{生僻字:垚瑄。}}此行一般可以看到内容,使用方正宋体\verb|\mystsong|命令,字体文件随模板一起提供和加载。

注意到这两种字体存在细微差异,因此一定要注意尽量仅在需要的时候使用\verb|\mystsong|,并且务必使用外层大括号全部括起来(内层的可以省略,使用空格分隔命令和内容本身),以确保环境范围的准确约束,如\verb|{\mystsong{生僻字:垚瑄}}|。

{\mystkaiti{为在各平台兼容生僻字,本模板中使用楷体的环境(如摘要、图表的标题等)已经默认设置为方正楷体。方正楷体的命令为\verb|\mystkaiti|,默认支持生僻字,如:垚瑄}}。


\textcolor{blue}{撰写引言部分,阅后删除}