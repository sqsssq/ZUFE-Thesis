% @Author: Qing Shi
% @LastEditTime: 2022/02/09

\documentclass[bachelor]{zufe}

\addbibresource{Reference.bib}

% % 使用中文编译 + XeLatex
% \usepackage[fontset=ubuntu]{ctex}

% 基本信息--------------------------------------------------------------------------------
% 在这里填写你的论文中文题目
\newcommand{\thesisTitle}{海岛环境对武学宗师成长的影响机理}
% 在这里填写你的论文英文题目
\newcommand{\thesisTitleEN}{Influence mechanism of island environment on the growth of martial arts masters}

% 若有副标题,则运行下一行的代码,若无副标题,则将下一行注释掉(在\haveSub{}最前面添加 % 号)
\haveSub{}

% 在这里填写你的论文中文副标题(没有只需注释掉\haveSub{}即可)
\newcommand{\thesisSubTitle}{基于桃花岛武学流派的研究}
% 在这里填写你的论文英文副标题(没有只需注释掉\haveSub{}即可)
\newcommand{\thesisSubTitleEN}{Research Based on Taohua island martial arts school}

% 在这里填写你的相关信息
\newcommand{\deptName}{信息管理与人工智能学院}
\newcommand{\majorName}{xxxx}
\newcommand{\yourName}{xx}
\newcommand{\yourStudentID}{180110910xxx}
\newcommand{\mentorName}{xxx}
\newcommand{\className}{xxx}
\newcommand{\Today}{2022年5月}
% 基本信息--------------------------------------------------------------------------------

% 文档开始
\begin{document}

% 封面,没有特殊情况不需要修改
% 封面
% 无特殊要求,不需要修改
% XelaTeX编译
\newcommand\dunderline[3][-1pt]{{%
  \setbox0=\hbox{#3}
  \ooalign{\copy0\cr\rule[\dimexpr#1-#2\relax]{\wd0}{#2}}}}

% Cover Page
\begin{titlepage}
  \makeatletter
  \@ifundefined{externalMentorName}{
    % 校内毕设封面顶部间距
    \vspace*{-20mm}
  }{
    % 校外毕设封面顶部间距
    \vspace*{13mm}
  }
  \centering

  \includegraphics[width=3cm]{InitFile/schoolLogo.png}\\
  \includegraphics[width=5cm, height=1cm]{InitFile/schoolName.png}

  \vspace*{10mm}
\begin{center}
  \zihao{1}\textbf{\ziju{0.12}\songti{本\hspace{5mm}科\hspace{5mm}生\hspace{5mm}毕\hspace{5mm}业\hspace{5mm}论\hspace{5mm}文(设计)}}

  \vspace{20mm}

  \xiaoer\textmd{\heiti{题目:}\thesisTitle}

  \vspace{5mm}
  \ifhaveSubTitle
  \begin{spacing}{1.2}
    \sanhao\selectfont{\textmd{\kaitigb{-----}\thesisSubTitle}}
  \end{spacing}
  \fi
  \vspace{30mm}

  \flushleft

  \makeatletter
  \@ifundefined{externalMentorName}{
    % 生成校内毕设封面字段
    \makeatother
    \begin{spacing}{1.8}
      \hspace{27mm}\songti\zihao{3}\selectfont{学生姓名:\dunderline[-10pt]{1pt}{\makebox[78mm][c]{\yourName}}}
      
      \hspace{27mm}\songti\zihao{3}\selectfont{学\hspace{11mm}号:\dunderline[-10pt]{1pt}{\makebox[78mm][c]{\yourStudentID}}}

      \hspace{27mm}\songti\zihao{3}\selectfont{指导教师:\dunderline[-10pt]{1pt}{\makebox[78mm][c]{\mentorName}}}
    
      \hspace{27mm}\songti\zihao{3}\selectfont{所在学院:\dunderline[-10pt]{1pt}{\makebox[78mm][c]{\deptName}}}

      \hspace{27mm}\songti\zihao{3}\selectfont{专业名称:\dunderline[-10pt]{1pt}{\makebox[78mm][c]{\majorName}}}
      
      \hspace{27mm}\songti\zihao{3}\selectfont{班\hspace{11mm}级:\dunderline[-10pt]{1pt}{\makebox[78mm][c]{\className}}}
    \end{spacing}
  }{
    % 生成校外毕设封面字段
    \makeatother
    \begin{spacing}{1.8}
      \hspace{19.4mm}\songti\zihao{3}\selectfont{学\hspace{19.6mm}院\hspace{3mm}:\dunderline[-10pt]{1pt}{\makebox[77.4mm][c]{\deptName}}}

      \hspace{19.4mm}\songti\zihao{3}\selectfont{专\hspace{19.6mm}业\hspace{3mm}:\dunderline[-10pt]{1pt}{\makebox[77.4mm][c]{\majorName}}}

      \hspace{19.4mm}\songti\zihao{3}\selectfont{学\hspace{2.8mm}生\hspace{2.8mm}姓\hspace{2.8mm}名\hspace{3mm}:\dunderline[-10pt]{1pt}{\makebox[77.4mm][c]{\yourName}}}

      \hspace{19.4mm}\songti\zihao{3}\selectfont{学\hspace{19.6mm}号\hspace{3mm}:\dunderline[-10pt]{1pt}{\makebox[77.4mm][c]{\yourStudentID}}}

      \hspace{19.4mm}\songti\zihao{3}\selectfont{指\hspace{2.8mm}导\hspace{2.8mm}教\hspace{2.8mm}师\hspace{3mm}:\dunderline[-10pt]{1pt}{\makebox[77.4mm][c]{\mentorName}}}

      \hspace{19.4mm}\songti\zihao{3}\selectfont{校外指导教师:\dunderline[-10pt]{1pt}{\makebox[77.4mm][c]{\externalMentorName}}}
    \end{spacing}
  }

  \vspace{25mm}
  \centering
  \zihao{4}\ziju{0.5}\xihei{\Today}
  \end{center}
\end{titlepage}



% 前置页面定义
\frontmatter

% 原创性声明,没有特殊情况不需要修改
\input{misc/1_originality}

% 摘要(中英文):根据自身论文,修改摘要的Tex文件

% \fancypagestyle{abstract}{
%   % 页眉高度
%   \setlength{\headheight}{10pt}

%   % 页眉和页脚(页码)的格式设定
%   \fancyhf{}
%   \fancyhead[C]{\ziju{0.08}\songti\zihao{-5}{浙江财经大学本科生毕业论文(设计)}}

%   % 页眉分割线稍微粗一些
%   \renewcommand{\headrulewidth}{0.4pt}
% }

% \pagestyle{abstract}
% \topskip=0pt

% 中英文摘要章节
\zihao{-4}
\vspace*{-10mm}

\begin{center}
  \heiti\zihao{3}\textmd{\thesisTitle}
  \vspace{2mm}
  \ifhaveSubTitle
  \begin{spacing}{1.2}
    \sihao\selectfont{\textmd{\mystkaiti{-----}\thesisSubTitle}}
  \end{spacing}
  \fi
\end{center}

\vspace*{0mm}

{\let\clearpage\relax \chapter*{\textmd{}}}
% 加入目录
\addcontentsline{toc}{chapter}{摘~~~~要}
\setcounter{page}{1}

\vspace*{-12mm}

\setstretch{1.53}
\setlength{\parskip}{0em}

\textbf{\heiti\xiaosi{摘要:}}
% 中文摘要正文从这里开始-----------------------------------------------------------------
{\mystkaiti{
本文……生僻字:垚瑄。{\songti{正文默认字体为\textbackslash{}songti命令,与支持生僻字的{\mystsong{方正宋体}}略有区别,因此正文中宋体生僻字使用\textbackslash{}mystsong命令代替即可。如:\mystsong{垚瑄}}}。
\textcolor{blue}{摘要的内容要包括研究的目的、方法、结果和结论。计量单位一律换算成国际标准计量单位。除特殊情况外,数字一律用阿拉伯数字。中、英文摘要的内容应严格一致。}
}}

\vspace{1em}
\textbf{\heiti\xiaosi{关键词:}}
{\mystkaiti{
计算机科学与技术;算法;复杂度;深度学习;机器学习;可视化计算
}}
% \newpage

% 英文摘要章节
\vspace*{15mm}

\begin{spacing}{0.95}
  \centering
  \zihao{3}\textbf{\thesisTitleEN}
  \vspace{2mm}
  \ifhaveSubTitle
  \begin{spacing}{1.2}
    \sihao\selectfont{\textmd{{-----}\thesisSubTitleEN}}
  \end{spacing}
  \fi
\end{spacing}

\vspace*{0mm}

{\let\clearpage\relax \chapter*{\zihao{-3}\textmd{}\vskip -3bp}}
\addcontentsline{toc}{chapter}{Abstract}
\setcounter{page}{1}

\setstretch{1.53}
\setlength{\parskip}{0em}

\textbf{\heiti\xiaosi{Abstract: }}
% 英文摘要正文从这里开始-----------------------------------------------------------------
In order to study……


\textbf{\xiaosi{Key words:}}
Computer science and technology; Algorithm; Complexity; Deep learning; Machine learning; Visualization analysis
\newpage


% 目录,自动生成,没有特殊情况不需要修改
% 论文目录
% 没有特殊需要不用修改

%目录开始

% 调整目录行间距
\renewcommand{\baselinestretch}{1.35}
% 目录
\tableofcontents
\newpage


% 正文开始
\mainmatter
% 正文 22 磅的行距
\setlength{\parskip}{0em}
\renewcommand{\baselinestretch}{1.53}
% 修复脚注出现跨页的问题
\interfootnotelinepenalty=10000

% 引言,根据自身论文,修改引言的Tex文件
% 引言

\chapter{引~~~~言}
本文主要研究了

\textcolor{blue}{撰写引言部分,阅后删除}

% 章节模板,成文后将其注释掉即可
\input{chapters/chapter_sample}

% 在此插入章节
% 第一章
% 第一章
% 根据自身论文修改

\chapter{相关工作}
人工智能算法是时代的热点,在国内外都有许多学者进行研究。
\section{国内研究现状}
\subsection{机器学习}
\subsection{深度学习}

\section{国外研究现状}
\subsection{机器学习}
\subsection{深度学习}
% 第二章,第三章······

% 结论:根据自身论文,修改结论的Tex文件
\input{misc/4_conclusion}

% 参考文献,无特殊要求不需要修改
% 添加参考文献请使用 BibTex 格式,添加至 Reference.bib 中,并在正文中使用 \cite{xxx}
% 成文后将\input{misc/5_reference}注释掉即可
\input{misc/5_reference}
\input{misc/5_simple_reference}

% 附录:根据自身论文,修改附录的Tex文件(补充论文,不是必须)
% 附录

\unnumchapter{附录A~~~~附录内容的名称}
\renewcommand{\thechapter}{附录}

% % 设置附录编号格式
% \ctexset{
%   section/number = 附录\Alph{section}
% }

% 附录相关内容…

% % 这里示范一下添加多个附录的方法:

% \section{\LaTeX 环境的安装}
% \LaTeX 环境的安装。

% \section{使用说明}
% 使用说明。

\textcolor{blue}{以下内容可放在附录之内:
1.正文内过于冗长的公式推导;
2.方便他人阅读所需的辅助性数学工具或表格;
3.重复性数据和图表;
4.论文使用的主要符号的意义和单位;
5.程序说明和程序全文;
6.调研报告。
}

\textcolor{blue}{这部分内容可省略(如果省略,删去此页)。阅后删除此段。}


% 致谢:根据自身要求,修改致谢的Tex文件
% 致谢

\unnumchapter{致~~~~谢}
\renewcommand{\thechapter}{致谢}

\ctexset{
  section/number = \arabic{section}
}

% 致谢部分尽量不使用 \subsection 二级标题,只使用 \section 一级标题

值此论文完成之际,首先向我的导师……

\textcolor{blue}{致谢正文样式与文章正文相同:宋体、小四;行距:22 磅;间距段前段后均为 0 行。阅后删除此段。}

\end{document}
